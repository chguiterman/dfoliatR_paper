\documentclass[review]{elsarticle} %review=doublespace preprint=single 5p=2 column
%%% Begin My package additions %%%%%%%%%%%%%%%%%%%
\usepackage[hyphens]{url}

  \journal{Tree-Ring Research} % Sets Journal name


\usepackage{lineno} % add
  \linenumbers % turns line numbering on
\providecommand{\tightlist}{%
  \setlength{\itemsep}{0pt}\setlength{\parskip}{0pt}}

\usepackage{graphicx}
\usepackage{booktabs} % book-quality tables
%%%%%%%%%%%%%%%% end my additions to header

\usepackage[T1]{fontenc}
\usepackage{lmodern}
\usepackage{amssymb,amsmath}
\usepackage{ifxetex,ifluatex}
\usepackage{fixltx2e} % provides \textsubscript
% use upquote if available, for straight quotes in verbatim environments
\IfFileExists{upquote.sty}{\usepackage{upquote}}{}
\ifnum 0\ifxetex 1\fi\ifluatex 1\fi=0 % if pdftex
  \usepackage[utf8]{inputenc}
\else % if luatex or xelatex
  \usepackage{fontspec}
  \ifxetex
    \usepackage{xltxtra,xunicode}
  \fi
  \defaultfontfeatures{Mapping=tex-text,Scale=MatchLowercase}
  \newcommand{\euro}{€}
\fi
% use microtype if available
\IfFileExists{microtype.sty}{\usepackage{microtype}}{}
\bibliographystyle{elsarticle-harv}
\usepackage{color}
\usepackage{fancyvrb}
\newcommand{\VerbBar}{|}
\newcommand{\VERB}{\Verb[commandchars=\\\{\}]}
\DefineVerbatimEnvironment{Highlighting}{Verbatim}{commandchars=\\\{\}}
% Add ',fontsize=\small' for more characters per line
\usepackage{framed}
\definecolor{shadecolor}{RGB}{248,248,248}
\newenvironment{Shaded}{\begin{snugshade}}{\end{snugshade}}
\newcommand{\AlertTok}[1]{\textcolor[rgb]{0.94,0.16,0.16}{#1}}
\newcommand{\AnnotationTok}[1]{\textcolor[rgb]{0.56,0.35,0.01}{\textbf{\textit{#1}}}}
\newcommand{\AttributeTok}[1]{\textcolor[rgb]{0.77,0.63,0.00}{#1}}
\newcommand{\BaseNTok}[1]{\textcolor[rgb]{0.00,0.00,0.81}{#1}}
\newcommand{\BuiltInTok}[1]{#1}
\newcommand{\CharTok}[1]{\textcolor[rgb]{0.31,0.60,0.02}{#1}}
\newcommand{\CommentTok}[1]{\textcolor[rgb]{0.56,0.35,0.01}{\textit{#1}}}
\newcommand{\CommentVarTok}[1]{\textcolor[rgb]{0.56,0.35,0.01}{\textbf{\textit{#1}}}}
\newcommand{\ConstantTok}[1]{\textcolor[rgb]{0.00,0.00,0.00}{#1}}
\newcommand{\ControlFlowTok}[1]{\textcolor[rgb]{0.13,0.29,0.53}{\textbf{#1}}}
\newcommand{\DataTypeTok}[1]{\textcolor[rgb]{0.13,0.29,0.53}{#1}}
\newcommand{\DecValTok}[1]{\textcolor[rgb]{0.00,0.00,0.81}{#1}}
\newcommand{\DocumentationTok}[1]{\textcolor[rgb]{0.56,0.35,0.01}{\textbf{\textit{#1}}}}
\newcommand{\ErrorTok}[1]{\textcolor[rgb]{0.64,0.00,0.00}{\textbf{#1}}}
\newcommand{\ExtensionTok}[1]{#1}
\newcommand{\FloatTok}[1]{\textcolor[rgb]{0.00,0.00,0.81}{#1}}
\newcommand{\FunctionTok}[1]{\textcolor[rgb]{0.00,0.00,0.00}{#1}}
\newcommand{\ImportTok}[1]{#1}
\newcommand{\InformationTok}[1]{\textcolor[rgb]{0.56,0.35,0.01}{\textbf{\textit{#1}}}}
\newcommand{\KeywordTok}[1]{\textcolor[rgb]{0.13,0.29,0.53}{\textbf{#1}}}
\newcommand{\NormalTok}[1]{#1}
\newcommand{\OperatorTok}[1]{\textcolor[rgb]{0.81,0.36,0.00}{\textbf{#1}}}
\newcommand{\OtherTok}[1]{\textcolor[rgb]{0.56,0.35,0.01}{#1}}
\newcommand{\PreprocessorTok}[1]{\textcolor[rgb]{0.56,0.35,0.01}{\textit{#1}}}
\newcommand{\RegionMarkerTok}[1]{#1}
\newcommand{\SpecialCharTok}[1]{\textcolor[rgb]{0.00,0.00,0.00}{#1}}
\newcommand{\SpecialStringTok}[1]{\textcolor[rgb]{0.31,0.60,0.02}{#1}}
\newcommand{\StringTok}[1]{\textcolor[rgb]{0.31,0.60,0.02}{#1}}
\newcommand{\VariableTok}[1]{\textcolor[rgb]{0.00,0.00,0.00}{#1}}
\newcommand{\VerbatimStringTok}[1]{\textcolor[rgb]{0.31,0.60,0.02}{#1}}
\newcommand{\WarningTok}[1]{\textcolor[rgb]{0.56,0.35,0.01}{\textbf{\textit{#1}}}}
\ifxetex
  \usepackage[setpagesize=false, % page size defined by xetex
              unicode=false, % unicode breaks when used with xetex
              xetex]{hyperref}
\else
  \usepackage[unicode=true]{hyperref}
\fi
\hypersetup{breaklinks=true,
            bookmarks=true,
            pdfauthor={},
            pdftitle={dfoliatR: An R package for detection and analysis of insect defoliation signals in tree rings},
            colorlinks=false,
            urlcolor=blue,
            linkcolor=magenta,
            pdfborder={0 0 0}}
\urlstyle{same}  % don't use monospace font for urls

\setcounter{secnumdepth}{0}
% Pandoc toggle for numbering sections (defaults to be off)
\setcounter{secnumdepth}{0}


% Pandoc header

\usepackage{amsmath}

\begin{document}
\begin{frontmatter}

  \title{\emph{dfoliatR}: An R package for detection and analysis of insect
defoliation signals in tree rings}
    \author[a,b]{Christopher H. Guiterman\corref{1}}
   \ead{chguiterman@email.arizona.edu} 
    \author[a,c]{Ann M. Lynch}
   \ead{lyncha@email.arizona.edu} 
    \author[c]{Jodi N. Axelson}
   \ead{jodi.axelson@berkeley.edu} 
      \address[a]{Laboratory of Tree-Ring Research, University of Arizona, 1215 E Lowell
St.~Box 210045, Tucson, AZ, 85721}
    \address[b]{Three Pines Forest Research, LLC, PO Box 225, Etna, NH, 03750}
    \address[c]{U.S. Forest Service, Rocky Mountain Research Station, 1215 E Lowell
St.~Box 210045, Tucson, AZ, 85721}
    \address[d]{Dept of Environmental Science, Policy \& Management, University of
California, Berkeley, Berkeley, CA 94720}
      \cortext[1]{Corresponding Author}
  
  \begin{abstract}
  We present a new R package to provide dendroecologists with tools to
  infer, quantify, analyze, and visualize growth suppression events in
  tree rings caused by insect defoliation. The `dfoliatR' library is based
  on the FORTRAN V program OUTBREAK, and builds on existing resources in
  the R computing environment. `dfoliatR' expands on OUTBREAK to provide
  greater control of suppression thresholds, additional output tables, and
  high-quality graphics. To use `dfoliatR' requires standardized
  ring-width measurements from insect host trees and an indexed tree-ring
  chronology from local non-host trees. It performs an indexing procedure
  to remove the climatic signal represented in the non-host chronology
  from the host-tree series. It then infers defoliation events in
  individual trees. Site-level analyses identify outbreak events that
  synchronously affect a user-defined number or proportion of the host
  trees. Functions are available for summary statistics and graphics of
  tree- and site-level series.
  
  \hfill\break
  \end{abstract}
   \begin{keyword} Dendroecology, spruce budworm, Choristoneura, pandora moth, Coloradia
pandora Blake, larch-bud-moth \newpage\end{keyword}
 \end{frontmatter}

\hypertarget{introduction}{%
\section{Introduction}\label{introduction}}

Variation in the width and morphology of annual radial growth rings in
wood permits dating and quantification of past forest insect defoliator
outbreaks. Defoliation can be distinguished from climate- and other
disturbance-related influences by comparing ring-width or other
annually-resolved features in the wood of host species to non-host
species or to climate records. The effect of defoliation on radial
growth of trees has been recognized since the mid-1800s, but work prior
to the 1980s was not cross-dated (based on annually-resolved and --dated
measurements), not standardized (quantitatively controlled for tree
geometry and age effects), and suffered from inadequate extraction of
the defoliation-related variability from variability associated with
climate and other factors (Swetnam et al. 1985, Speer 2010, Lynch 2012).
These and other problems may be circumvented by appropriate use of
dendrochronological techniques. Quantitative methodology to infer forest
defoliator outbreaks from cross-dated tree-ring records was developed in
the 1980s by Swetnam et al. (1985) for developing western spruce budworm
chronologies (Swetnam and Lynch 1989, 1993). The methodology has since
been successfully applied to a wide range of defoliator species, most of
which are conifer herbivores, and has evolved in sophistication and
application to a wide range of ecosystem situations (Lynch 2012).

The main dendrochronological tool for inferring, dating, and quantifying
defoliator outbreaks from tree-ring records has been the software
routine OUTBREAK (Swetnam et al. 1985, Holmes and Swetnam 1986, Swetnam
and Lynch 1989). OUTBREAK computes indices (described later in detail)
of suppressed growth by subtracting a detrended and standardized climate
series (a ``control'' chronology, usually a site chronology developed
from non-host trees or a gridded climate data point series) from host
individual-tree detrended and standardized radial growth series after
the host and non-host series have been brought to a common variance. If
the host and non-host species respond similarly to climate (which can be
tested), the derived series retains variability that the host and
non-host series do not have in common, generally the insect signal and
noise (unexplained variability). The user defines a rule base specifying
the magnitude and duration that a period of indexed growth suppression
must meet or surpass for a period of suppressed growth to be inferred as
a defoliation event at the tree level.

Though powerful, OUTBREAK is outdated and increasingly difficult to use
in modern computing environments. It was written in Fortran V with
inherently severe restrictions, as RAM and disk space were limited at
that time (256 kb and 10 MB, respectively) and Fortran conventions
imposed very strict formatting, file naming, and output conventions.
Windows-based execution operates in a DOS window, Apple-based xxx in
xxx, and provides no graphical interface or capabilities. Barriers to
batch operation impose burdens for analyses of larger data sets. We
developed dfoliaR as an \texttt{R}- and \texttt{dplR}-based routine to
overcome these issues.

\texttt{dfoliatR} adds to a growing suite of dendrochronology packages
the \texttt{R} computing environment (R Core Team 2019). Stemming from
the \texttt{dplR} library (Bunn 2008) that enables \texttt{R} users to
read and write an array of tree-ring data formats, standardize ring
width series, build and evaluate chronologies, and perform quality
control (to name a few), one can now also measure ring widths from
scanned images of preparred samples (Lara et al. 2015, Shi and Xiang
2019), perform and check crossdating (Bunn 2010), and perform many
analytical tests (Zang and Biondi 2015, Jevšenak and Levanič 2018).
Tools for assessing stand dynamics and disturbance analyses are under
rapid development, with new packages for assessing growth and release
events (\texttt{TRADER}: Altman et al. 2014), metrics of growth
resilience (\texttt{pointRes}: Maaten-Theunissen et al. 2015), and fire
history (\texttt{burnr}: Malevich et al. 2018). The key objective of
\texttt{dfoliatR} is to provide tools to identify and analyze insect
defoliation and outbreak events by building on the methods employed by
OUTBREAK. It capitilizes on the robust software already available in
\texttt{R} by using \texttt{dplR} data formats for incoming tree-ring
series and providing output data formats embodied by the
\texttt{tidyverse} (Wickham et al. 2019) that include efficient data
manipulation (\texttt{dplyr}: Wickham et al. 2020) and graphics
(\texttt{ggplot2}: Wickham 2016).

In this paper, we describe the statistical methods employed by
\texttt{dfoliatR}, its availability, compare results to those produced
by OUTBREAK, and present an example analyses. Users need not have much
experience in \texttt{R} to replicate the analyses and graphics as
presented. The \texttt{R} code presented below is executable in an
\texttt{R} session once the required libraries are installed and loaded.
Support documentation in addition to this paper is provided within the
package via standard help menus (accessed by typing \texttt{?} before a
function name) and on the package website
(\url{https://chguiterman.github.io/dfoliatR/}), which includes
up-to-date vignettes that describe in detail the functionality of the
software. Code to create a preprint of this manuscript including the
\texttt{R} scripts is available from
\url{https://github.com/chguiterman/dfoliatR_paper}.

\hypertarget{overview-of-the-software}{%
\section{Overview of the software}\label{overview-of-the-software}}

The \texttt{dfoliatR} library requires two sets of tree-ring data to
infer defoliation and outbreak events:

\begin{itemize}
\tightlist
\item
  Standardized ring-width series for individual trees of the host
  species
\item
  Standardized tree-ring chronology from a local non-host species
\end{itemize}

Users can develop these data sets in software of their choosing, such as
\texttt{dplR} (Bunn 2008) or ARSTAN (Cook and Holmes 1996). It is
important that the host-tree data include only one tree-ring series per
tree. Both \texttt{dplR} and ARSTAN have options for averaging multiple
sample series into a tree-level series.

At the heart of \texttt{dfoliatR} lies two functions:
\texttt{defoliate\_trees()} and \texttt{outbreak()}. These identify
defoliation events on individual trees and then composite across
multiple trees to identify outbreak events.

\hypertarget{identifying-defoliation-of-trees}{%
\subsection{Identifying Defoliation of
Trees}\label{identifying-defoliation-of-trees}}

The \texttt{defoliate\_trees()} function is usually be the point of
entry to the dfoliatR library. It performs two processes, removing
climate-related growth signals from the host-tree series and then
identifying tree-level defoliation events. The climatic or
non-defoliation signals in each host-tree series are characterized by a
non-host chronology or climate reconstruction. \texttt{dfoiatR} removes
that non-defoliation signal by subtracting the non-host series from each
host-tree series, which generates a residual index. In OUTBREAK, this
residual index was termed the ``corrected'' index. We call it the
``growth suppression index'' (GSI). The GSI is calculated the same as in
OUTBREAK (following Swetnam et al. 1985, Swetnam and Lynch 1989) for
each host tree as \begin{align}
\textrm{GSI}_{i} = \textrm{H}_{i} - \left( \textrm{NH}_{i} - \overline{\textrm{NH}} \right) \frac{\sigma_{\textrm{H}}}{\sigma_{\textrm{NH}}} \
\end{align} where H and NH are the host-tree series and the non-host
chronology, in year i, respectively. Only the common period between the
host-tree series and the non-host chronology are used in Equation 1. The
host and non-host chronologies are brought to common variance by scaling
the non-host chronology by its mean
(\(\overline{\textrm{NH}} \approx 1.0\)) and multiplying by the ratio of
host and non-host standard deviations
(\(\frac{\sigma_{\textrm{H}}}{\sigma_{\textrm{NH}}}\)), which
approximates the variance of the host tree series.

Negative departures in the normalized GSI that surpass user-defined
thresholds in duration and magnitude are defined as \emph{defoliation
events}. As in OUTBREAK, magnitude is assessed on a single year within
the departure sequence. The default setting is -1.28 (in units of
standard deviation), which was previously determined to be
representative of WSBW effects (\textbf{citation?}). Duration is
assessed by examining sequences of negative GSI before and after the
year of maximum departure. Each defoliation event is allowed one
positive excursion on each side of the maximum departure year. Duration
is assessed across the entire sequence that includes up to two positive
excursions. The default duration is eight years, as is commonly used in
WSBW studies (\textbf{citation?}). Different species of defoliation
insects vary in the length of defoliation and the degree to which they
can suppress tree growth. Researchers can, and should, adjust the
duration and magnitude parameters accordingly and critically evaluate
the results.

Diverging from OUTBREAK, \texttt{dfoliatR} allows users to extend
defoliation events by bridging successive events and also by allowing
potentially short-duration events that occur at the end of the series.
In cases where two defoliation events are separated by a single year,
bridging will link them into a single event (\textbf{Show figure?}). We
urge careful use of this option because there is no setting to limit the
number or length of potentially bridged events. The series end option
can be used in cases when the host trees were actively being defoliated
at the time of sampling. This option eliminates the duration parameter
for an event at the recent end of the series, but all other thresholds
apply. The advantage of this parameter is that it can aid in identifying
the start-year for the current defoliation event or outbreak, which is
both useful in management and allows the current event to be included in
return-interval estimates.

\hypertarget{inferring-outbreak-events}{%
\subsection{Inferring Outbreak Events}\label{inferring-outbreak-events}}

Defoliation of one or a few trees does not constitute an outbreak. To
determine when defoliation becomes an \emph{outbreak event},
\texttt{dfoliatR} composites the individual tree defoliation series into
a site-level chronology with the \texttt{outbreak()} function. Users
have options to define the number and/or the proportion of trees
required for an event to be considered an outbreak. Three parameters
control the whether a defoliation event constitutes an outbreak: the
minimum number of trees available, the minimum number of trees recording
defoliation, and the percent of trees recording defoliation. The first
allows the researcher to make a judgement call as to the confidence
ascribed to reduced sample depth toward the ends of their chronologies,
thus compensating for the ``fading record problem'' (\textbf{Swetnam and
Fritts?}). The second two parameters adjust the scale of defoliation
considered to be an outbreak. Absolute numbers of trees and percentages
can be applied separately or in conjunction, following filtering
conventions in tree ring fire history studies (Malevich et al. 2018). We
urge users to carefully consider the choice of absolute numbers in
situations where the number of trees represented in the series varies
with time, or the choice of percentages when sample size is small.

\hypertarget{evaluation}{%
\section{Evaluation}\label{evaluation}}

\hypertarget{approach}{%
\subsection{Approach}\label{approach}}

We tested \texttt{dfoliatR} against OUTBREAK by comparing GSI to
corrected indices for individual trees and years, outbreak status for
individual trees and years, and percentage of trees recording outbreaks
at the site level, using raw ring-width data from 8 sites in British
Columbia, Colorado, and New Mexico and author-provided non-host site
chronologies.

We detrended host data for both dfoliatR and OUTBREAK using ARSTAN 6.1
(\textbf{downloaded 21 April 2002 from dpl; xxx Ann check the date, as
21 April 2002 is 6.05P}) with default double detrending (128 year
wavelength and a 50\% smoothing spline) and event thresholds of -1.28
normalized index and 8 years. Attempts to detrend with ARSTAN 44 in
\texttt{dplR} were problematic for two reasons. Some versions of ARSTAN,
including ARSTAN 44, do not include an option for producing tree-level
averages from multiple cores from the same tree (.tre files). More
importantly, ring-width indices from ARSTAN 44 truncated the process at
either end when sample size fell below 5 xxx (Jodi, what was the
minimum?), and produced different values for years near the truncation
point than did dpl-ARSTAN 6.05P. The majority of the detrended series
values were identical, but not the tails, as the two routines do not use
the same data to establish the tails. These differences produce
different GSI (\texttt{dfoliatR}) and Corrected Indices (OUTBREAK)
values at the tails, sometimes considerably, affecting whether or not
outbreak events are inferred early or late in the tree series, timing of
the first or last event onset, and subsequent computation of return
intervals, duration, and periodicity. Note that there is nothing
particularly important regarding the choice of ARSTAN 6.1 -- it is
simply the version that the second author used routinely and had
available.

\hypertarget{findings}{%
\subsection{Findings}\label{findings}}

\texttt{dfoliatR} and OUTBREAK compute growth suppresses indices and use
them to infer identical tree-level outbreak events, including onset and
termination dates and date and magnitude of the maximum growth
suppression when \texttt{dfoliatR} bridging matches the OUTBREAK
protocols. Once the issues with ARSTAN (described earlier) were
resolved, dfoliatR and OUTBREAK produced identical indices at 0.000
precision.

When inferring outbreak events at the tree level, our initial results
did not always match OUTBREAK, leading to development of the bridging
option. Identifying a minimum within a period while allowing for
positive excursions is a tricky computation. OUTBREAK deliberately
disallows back-to-back events -- after a minimum value within a period
of negative values, a second positive index terminates an event.
OUTBREAK does not allow a period of highly negative values to be
appended to an earlier outbreak if the intervening period is 1 year,
even if those negative values fall below the minimum in the earlier
period. Due in large part to reconstructions using OUTBREAK (see papers
cited by Lynch 2012), we now know considerably more about forest
defoliator outbreak regimes than we did in the 1980s when OUTBREAK was
written. We think that two prolonged events separated by a single year
should in some situations be considered a single outbreak at both the
tree and site levels. This is particularly relevant to western spruce
budworm and spruce budworm (\emph{C. fumiferana}), for which regimes
have been reconstructed for several geographic areas and for which
researchers and forest health experts have gained considerable knowledge
(\textbf{Sanders et al.~1985, Brookes et al.~1987 and many later
publications}). We now know that the greatest growth suppression often
occurs late in the outbreak due to cumulative effects (\textbf{rrr}). We
developed the bridging option to permit these periods to be linked to
preceding events separated by a single year.

Bridging operates at the tree level in dfoliatR, and these effects
generally have minor effects on site-level reconstructions
(Fig.Bridging). Occasionally site-level events are inferred differently
(Table.Bridging), and this can affect return interval and duration
statistics.

\hypertarget{best-practice-recommendations}{%
\subsection{Best Practice
Recommendations}\label{best-practice-recommendations}}

We encourage authors to report software and library version numbers and
download dates where applicable. This is standard practice for
statistical packages, but has often been ignored for R- and dplR-based
analyses. R-based software evolves over time, and results cannot always
be replicated if the software versions are unknown. Similarly,
significant changes (especially fixes) imply that re-analysis of data
might differ from earlier published results and interpretations.
Wherever possible, refereed publications supporting the software should
be cited. Our experience with ARSTAN demonstrates the importance of
adopting this practice. In the United States, analytic methods not
documented thusly and supported by the literature might be questioned if
the results are used in a National Environmental Protection Act (NEPA)
process.

We recommend that researchers compare dfoliatR results with and without
bridging, and see which option agrees best with known insect biology and
ecology. Bridging may be a logical choice for insects with prolonged
non-outbreak periods, such as western spruce budworm, but not for
situations where impacted stands barely recover from one outbreak before
another begins, such as occurs with pine processionary caterpillars
(\emph{Thaumetopoea pityocampa} (Lepidoptera: Thaumetopoeidae))
(\textbf{Carus 2004, 2009}).

\hypertarget{availability-and-installation}{%
\section{Availability and
installation}\label{availability-and-installation}}

The \texttt{dfoliatR} library (Guiterman et al. 2020) is provided free
and open source from the Comprehensive R Archive Network (CRAN;
\url{https://cran.r-project.org/}). To install \texttt{dfoliatR} from
CRAN use

\begin{Shaded}
\begin{Highlighting}[]
\KeywordTok{install.packages}\NormalTok{(}\StringTok{"dfoliatR"}\NormalTok{)}
\end{Highlighting}
\end{Shaded}

In each \texttt{R} session, \texttt{dfoliatR} can be loaded via

\begin{Shaded}
\begin{Highlighting}[]
\KeywordTok{library}\NormalTok{(dfoliatR)}
\end{Highlighting}
\end{Shaded}

Development versions of \texttt{dfoliatR} are available on GitHub and
installed using the \texttt{devtools} library,

\begin{Shaded}
\begin{Highlighting}[]
\NormalTok{devtools}\OperatorTok{::}\KeywordTok{install_github}\NormalTok{(}\StringTok{"chguiterman/dfoliatR"}\NormalTok{)}
\end{Highlighting}
\end{Shaded}

Issues, bug reports, and ideas for improving \texttt{dfoliatR} can be
posted to \url{https://github.com/chguiterman/dfoliatR/issues}. As an
Open Source library, we welcome and encourage community involvement in
future development. The best ways to contribute to \texttt{dfoliatR} are
through standard GitHub procedures or by contacting the first author.

\hypertarget{example-usage}{%
\section{Example Usage}\label{example-usage}}

In \texttt{dfoliatR} we provide two sets of tree-ring data to aid users
in exploring the functions, graphics, and outputs. Each set consists of
Douglas-fir (\emph{Pseudotsuga menziesii}) host-tree series,
standardized with \_\_\_\_\_\_\_\_\_\_, and a local ponderosa pine
(\emph{Pinus ponderosa}) non-host chronology. The non-host ring-width
data were standardized by \_\_\_\_\_\_\_\_ and the chronologies averaged
following standard procedures (Speer 2010). Data from Demijohn Peak
(DMJ; 2902 m asl), in the San Juan Mountains of southern Colorado, come
from Ryerson et al. (2003). Data from the East Fork site (EF; 2580 m
asl) in the Jemez Mountains of northcentral New Mexico were presented by
Swetnam and Lynch (1993).

\hypertarget{tree-level-defoliation-events}{%
\subsection{Tree-Level Defoliation
Events}\label{tree-level-defoliation-events}}

\hypertarget{site-level-events}{%
\subsection{Site-Level Events}\label{site-level-events}}

\hypertarget{extensions}{%
\section{Extensions}\label{extensions}}

\begin{itemize}
\tightlist
\item
  Describe how dfoliatR can be combined with other R libraries

  \begin{itemize}
  \tightlist
  \item
    Mapping, what else?
  \end{itemize}
\end{itemize}

\hypertarget{references}{%
\section*{References}\label{references}}
\addcontentsline{toc}{section}{References}

\hypertarget{refs}{}
\leavevmode\hypertarget{ref-Altman2014}{}%
Altman, J., P. Fibich, J. Dolezal, and T. Aakala. 2014. TRADER: A
package for Tree Ring Analysis of Disturbance Events in R.
Dendrochronologia 32:107--112.

\leavevmode\hypertarget{ref-Bunn2008}{}%
Bunn, A. G. 2008. A dendrochronology program library in R (dplR).
Dendrochronologia 26:115--124.

\leavevmode\hypertarget{ref-Bunn2010}{}%
Bunn, A. G. 2010. Statistical and visual crossdating in R using the dplR
library. Dendrochronologia 28:251--258.

\leavevmode\hypertarget{ref-arstan}{}%
Cook, E. R., and R. L. Holmes. 1996. Guide for computer program arstan.
Pages 75--87 The international tree-ring data bank program library
version.

\leavevmode\hypertarget{ref-dfoliatR}{}%
Guiterman, C., A. Lynch, and J. Axelson. 2020. DfoliatR: Detection and
analysis of insect defoliation signals in tree rings.

\leavevmode\hypertarget{ref-outbreak}{}%
Holmes, R. L., and T. W. Swetnam. 1986. Dendroecology program library:
Program outbreak user's manual. Laboratory of Tree-Ring Research,
University of Arizona, Tucson.

\leavevmode\hypertarget{ref-Jevsenak2018}{}%
Jevšenak, J., and T. Levanič. 2018. dendroTools: R package for studying
linear and nonlinear responses between tree-rings and daily
environmental data. Dendrochronologia 48:32--39.

\leavevmode\hypertarget{ref-Lara2015}{}%
Lara, W., F. Bravo, and C. A. Sierra. 2015. MeasuRing: An R package to
measure tree-ring widths from scanned images. Dendrochronologia
34:43--50.

\leavevmode\hypertarget{ref-Lynch2012}{}%
Lynch, A. M. 2012. What Tree-Ring Reconstruction Tells Us about Conifer
Defoliator Outbreaks. Pages 126--154 \emph{in} P. Barbosa, D. K.
Letourneau, and A. A. Agrawal, editors. Insect outbreaks revisited.
Blackwell Publishing Ltd.

\leavevmode\hypertarget{ref-VanderMaaten-Theunissen2015}{}%
Maaten-Theunissen, M. van der, E. van der Maaten, and O. Bouriaud. 2015.
PointRes: An R package to analyze pointer years and components of
resilience. Dendrochronologia 35:34--38.

\leavevmode\hypertarget{ref-Malevich2018}{}%
Malevich, S. B., C. H. Guiterman, and E. Q. Margolis. 2018. burnr: Fire
history analysis and graphics in R. Dendrochronologia 49:9--15.

\leavevmode\hypertarget{ref-RCore}{}%
R Core Team. 2019. R: A language and environment for statistical
computing. R Foundation for Statistical Computing, Vienna, Austria.

\leavevmode\hypertarget{ref-Ryerson2003}{}%
Ryerson, D. E., T. W. Swetnam, and A. M. Lynch. 2003. A tree-ring
reconstruction of western spruce budworm outbreaks in the San Juan
Mountains, Colorado, U.S.A. Canadian Journal of Forest Research
33:1010--1028.

\leavevmode\hypertarget{ref-Shi2019}{}%
Shi, J., and W. Xiang. 2019. MtreeRing: A shiny application for
automatic measurements of tree-ring widths on digital images.

\leavevmode\hypertarget{ref-Speer2010}{}%
Speer, J. H. 2010. Fundamentals of tree-ring research. Page 333. The
University of Arizona Press.

\leavevmode\hypertarget{ref-Swetnam1989}{}%
Swetnam, T. W., and A. M. Lynch. 1989. A tree-ring reconstruction of
western spruce budworm history in the southern Rocky Mountains. Forest
Science 35:962--986.

\leavevmode\hypertarget{ref-Swetnam1993}{}%
Swetnam, T. W., and A. M. Lynch. 1993. Multicentury, Regional-Scale
Patterns of Western Spruce Budworm Outbreaks. Ecological Monographs
63:399--424.

\leavevmode\hypertarget{ref-Swetnam1985}{}%
Swetnam, T. W., M. A. Thompson, and E. K. Sutherland. 1985. Using
dendrochronology to measure radial growth of defoliated trees. Page 39p.

\leavevmode\hypertarget{ref-wickham2016ggplot2}{}%
Wickham, H. 2016. Ggplot2: Elegant graphics for data analysis. Springer.

\leavevmode\hypertarget{ref-Wickham2019}{}%
Wickham, H., M. Averick, J. Bryan, W. Chang, L. McGowan, R. François, G.
Grolemund, A. Hayes, L. Henry, J. Hester, M. Kuhn, T. Pedersen, E.
Miller, S. Bache, K. Müller, J. Ooms, D. Robinson, D. Seidel, V. Spinu,
K. Takahashi, D. Vaughan, C. Wilke, K. Woo, and H. Yutani. 2019. Welcome
to the tidyverse. Journal of Open Source Software 4:1686.

\leavevmode\hypertarget{ref-Wickham2020dplyr}{}%
Wickham, H., R. François, L. Henry, and K. Müller. 2020. Dplyr: A
grammar of data manipulation.

\leavevmode\hypertarget{ref-Zang2015}{}%
Zang, C., and F. Biondi. 2015. Treeclim: An R package for the numerical
calibration of proxy-climate relationships. Ecography 38:431--436.


\end{document}


