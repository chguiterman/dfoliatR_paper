\documentclass[review]{elsarticle} %review=doublespace preprint=single 5p=2 column
%%% Begin My package additions %%%%%%%%%%%%%%%%%%%
\usepackage[hyphens]{url}

  \journal{Tree-Ring Research} % Sets Journal name


\usepackage{lineno} % add
  \linenumbers % turns line numbering on
\providecommand{\tightlist}{%
  \setlength{\itemsep}{0pt}\setlength{\parskip}{0pt}}

\usepackage{graphicx}
\usepackage{booktabs} % book-quality tables
%%%%%%%%%%%%%%%% end my additions to header

\usepackage[T1]{fontenc}
\usepackage{lmodern}
\usepackage{amssymb,amsmath}
\usepackage{ifxetex,ifluatex}
\usepackage{fixltx2e} % provides \textsubscript
% use upquote if available, for straight quotes in verbatim environments
\IfFileExists{upquote.sty}{\usepackage{upquote}}{}
\ifnum 0\ifxetex 1\fi\ifluatex 1\fi=0 % if pdftex
  \usepackage[utf8]{inputenc}
\else % if luatex or xelatex
  \usepackage{fontspec}
  \ifxetex
    \usepackage{xltxtra,xunicode}
  \fi
  \defaultfontfeatures{Mapping=tex-text,Scale=MatchLowercase}
  \newcommand{\euro}{€}
\fi
% use microtype if available
\IfFileExists{microtype.sty}{\usepackage{microtype}}{}
\bibliographystyle{elsarticle-harv}
\usepackage{color}
\usepackage{fancyvrb}
\newcommand{\VerbBar}{|}
\newcommand{\VERB}{\Verb[commandchars=\\\{\}]}
\DefineVerbatimEnvironment{Highlighting}{Verbatim}{commandchars=\\\{\}}
% Add ',fontsize=\small' for more characters per line
\usepackage{framed}
\definecolor{shadecolor}{RGB}{248,248,248}
\newenvironment{Shaded}{\begin{snugshade}}{\end{snugshade}}
\newcommand{\AlertTok}[1]{\textcolor[rgb]{0.94,0.16,0.16}{#1}}
\newcommand{\AnnotationTok}[1]{\textcolor[rgb]{0.56,0.35,0.01}{\textbf{\textit{#1}}}}
\newcommand{\AttributeTok}[1]{\textcolor[rgb]{0.77,0.63,0.00}{#1}}
\newcommand{\BaseNTok}[1]{\textcolor[rgb]{0.00,0.00,0.81}{#1}}
\newcommand{\BuiltInTok}[1]{#1}
\newcommand{\CharTok}[1]{\textcolor[rgb]{0.31,0.60,0.02}{#1}}
\newcommand{\CommentTok}[1]{\textcolor[rgb]{0.56,0.35,0.01}{\textit{#1}}}
\newcommand{\CommentVarTok}[1]{\textcolor[rgb]{0.56,0.35,0.01}{\textbf{\textit{#1}}}}
\newcommand{\ConstantTok}[1]{\textcolor[rgb]{0.00,0.00,0.00}{#1}}
\newcommand{\ControlFlowTok}[1]{\textcolor[rgb]{0.13,0.29,0.53}{\textbf{#1}}}
\newcommand{\DataTypeTok}[1]{\textcolor[rgb]{0.13,0.29,0.53}{#1}}
\newcommand{\DecValTok}[1]{\textcolor[rgb]{0.00,0.00,0.81}{#1}}
\newcommand{\DocumentationTok}[1]{\textcolor[rgb]{0.56,0.35,0.01}{\textbf{\textit{#1}}}}
\newcommand{\ErrorTok}[1]{\textcolor[rgb]{0.64,0.00,0.00}{\textbf{#1}}}
\newcommand{\ExtensionTok}[1]{#1}
\newcommand{\FloatTok}[1]{\textcolor[rgb]{0.00,0.00,0.81}{#1}}
\newcommand{\FunctionTok}[1]{\textcolor[rgb]{0.00,0.00,0.00}{#1}}
\newcommand{\ImportTok}[1]{#1}
\newcommand{\InformationTok}[1]{\textcolor[rgb]{0.56,0.35,0.01}{\textbf{\textit{#1}}}}
\newcommand{\KeywordTok}[1]{\textcolor[rgb]{0.13,0.29,0.53}{\textbf{#1}}}
\newcommand{\NormalTok}[1]{#1}
\newcommand{\OperatorTok}[1]{\textcolor[rgb]{0.81,0.36,0.00}{\textbf{#1}}}
\newcommand{\OtherTok}[1]{\textcolor[rgb]{0.56,0.35,0.01}{#1}}
\newcommand{\PreprocessorTok}[1]{\textcolor[rgb]{0.56,0.35,0.01}{\textit{#1}}}
\newcommand{\RegionMarkerTok}[1]{#1}
\newcommand{\SpecialCharTok}[1]{\textcolor[rgb]{0.00,0.00,0.00}{#1}}
\newcommand{\SpecialStringTok}[1]{\textcolor[rgb]{0.31,0.60,0.02}{#1}}
\newcommand{\StringTok}[1]{\textcolor[rgb]{0.31,0.60,0.02}{#1}}
\newcommand{\VariableTok}[1]{\textcolor[rgb]{0.00,0.00,0.00}{#1}}
\newcommand{\VerbatimStringTok}[1]{\textcolor[rgb]{0.31,0.60,0.02}{#1}}
\newcommand{\WarningTok}[1]{\textcolor[rgb]{0.56,0.35,0.01}{\textbf{\textit{#1}}}}
\ifxetex
  \usepackage[setpagesize=false, % page size defined by xetex
              unicode=false, % unicode breaks when used with xetex
              xetex]{hyperref}
\else
  \usepackage[unicode=true]{hyperref}
\fi
\hypersetup{breaklinks=true,
            bookmarks=true,
            pdfauthor={},
            pdftitle={dfoliatR: An R package for detection and analysis of insect defoliation signals in tree rings},
            colorlinks=false,
            urlcolor=blue,
            linkcolor=magenta,
            pdfborder={0 0 0}}
\urlstyle{same}  % don't use monospace font for urls

\setcounter{secnumdepth}{0}
% Pandoc toggle for numbering sections (defaults to be off)
\setcounter{secnumdepth}{0}


% Pandoc header

\usepackage{amsmath}

\begin{document}
\begin{frontmatter}

  \title{\emph{dfoliatR}: An R package for detection and analysis of insect
defoliation signals in tree rings}
    \author[a,b]{Christopher H. Guiterman\corref{1}}
   \ead{chguiterman@email.arizona.edu} 
    \author[a,c]{Ann M. Lynch}
   \ead{lyncha@email.arizona.edu} 
    \author[c]{Jodi N. Axelson}
   \ead{jodi.axelson@berkeley.edu} 
      \address[a]{Laboratory of Tree-Ring Research, University of Arizona, 1215 E Lowell
St.~Box 210045, Tucson, AZ, 85721}
    \address[b]{Three Pines Forest Research, LLC, PO Box 225, Etna, NH, 03750}
    \address[c]{U.S. Forest Service, Rocky Mountain Research Station, 1215 E Lowell
St.~Box 210045, Tucson, AZ, 85721}
    \address[d]{Dept of Environmental Science, Policy \& Management, University of
California, Berkeley, Berkeley, CA 94720}
      \cortext[1]{Corresponding Author}
  
  \begin{abstract}
  We present a new R package to provide dendroecologists with tools to
  identify, quantify, analyze, and visualize growth suppression events in
  tree rings produced by insect defoliation. The `dfoliatR' library is
  based on the Fortran V program OUTBREAK, and builds on existing
  resources in the R computing environment. `dfoliatR' expands on OUTBREAK
  to provide greater control of supression thresholds, additional output
  tables, and high-quality graphics. To use `dfoliatR' requires
  standardized ring-width measurements from insect host trees and an
  indexed tree-ring chronology from local non-host trees. It performs an
  indexing procedure to remove the climatic signal represented in the
  non-host chronology from the host-tree series. It then infers
  defoliation events in individual trees. Site-level analyses identify
  outbreak events that synchronously affect a user-defined number or
  proportion of the host trees. Functions are available for summary
  statistics and graphics of tree- and site-level series.
  
  \hfill\break
  \end{abstract}
   \begin{keyword} Dendroecology, spruce budworm, Choristoneura, pandora moth, Coloradia
pandora Blake, larch-bud-moth \newpage\end{keyword}
 \end{frontmatter}

\hypertarget{introduction}{%
\section{Introduction}\label{introduction}}

The \texttt{dfoliatR} library is unique among a growing suite \texttt{R}
packages designed for dendrochronology. Stemming from the \texttt{dplR}
library (Bunn 2008) that provides \texttt{R} the ability to read and
write an array of tree-ring data formats, standaridize ring widths,
build and evaluate chronologies, perform qaulity control (to name a
few), one can now also measure, perform and check crossdating, and
perform many analytical tests (Bunn 2010, Lara et al. 2015, Zang and
Biondi 2015, Jevšenak and Levanič 2018). Tools for assessing stand
dynamics and disturbance analyses are under rapid development, with new
packages for assessing growth and release events (\texttt{TRADER}:
Altman et al. 2014), metrics of growth resilience (\texttt{pointRes}:
Maaten-Theunissen et al. 2015), and fire history (\texttt{burnr}:
Malevich et al. 2018). The key objective of \texttt{dfoliatR} is to
provide tools to identify and analyse insect defoliation and outbreak
events by building on the methods employed the FORTRAN program OUTBREAK
(Swetnam and Lynch 1989). What sets \texttt{dfoliatR} apart from
packages such as \texttt{TRADER} is that it explicitly performs an
indexing procedue on host-tree series to remove climatic and other
non-defoliation related signals represented by non-host tree species.
Insect defoliation signals are identified in the disturbance index by
the duration and magnitude of negative departures.

\texttt{dfoliatR} draws upon data formats in \texttt{dplR} that are
commonly employed by other tree-ring libraries. It uses and outputs a
data formats that faciliate the use packages embodied by
\texttt{tidyverse} (Wickham et al. 2019) that include efficient data
wrangling (\texttt{dplyr}: Wickham et al. 2020) and graphics
(\texttt{ggplot2}: Wickham 2016).

In this paper, we describe the statistical methods employed by
\texttt{dfoliatR}, its availability, and run through analyses for a
signle site in New Mexico. Users need not have much experience in
\texttt{R} to replicate the analyses and graphics below. All \texttt{R}
code presented below is executatable in an \texttt{R} session once the
required libraries are installed and loaded. Support documentation in
addition to this paper is provided within the package via standard help
menus (accessed by typing \texttt{?} before a function name) and on the
package website (\url{https://chguiterman.github.io/dfoliatR/}), which
includes up-to-date vignettes that describe in detail the functionality
of the software. Code to create a preprint of this manuscript including
the \texttt{R} scripts is available from
\url{https://github.com/chguiterman/dfoliatR_paper}.

\hypertarget{overview-of-the-software}{%
\section{Overview of the software}\label{overview-of-the-software}}

The \texttt{dfoliatR} library requires two sets of tree-ring data to
identify defoliation and outbreak events:

\begin{itemize}
\tightlist
\item
  Standardized ring-width series for individual trees of the host
  species
\item
  Standardized tree-ring chronology from a local non-host species
\end{itemize}

Users can develop these data sets in software of their choosing, such as
in \texttt{dplR} or ARSTAN. It is important that the host-tree data
include only one tree-ring series per tree. Both \texttt{dplR} and
ARSTAN have options for averaging mutliple sample series into a
tree-level series. The tree-ring series and chronology can be read into
\texttt{R} via several available \texttt{dplR} functions.

\texttt{dfoliatR} begins to identify defoliation events in individual
trees by removing the climatic signal as represented by the non-host
chronology from the host tree series. This indexing procedure creates a
series -- the ``growth suppression index'' (GSI) -- in which disturbance
is the predominant signal. The GSI is caluclated as \begin{align}
\textrm{GSI}_{i} = \textrm{H}_{i} - \left( \textrm{NH}_{i} - \overline{\textrm{NH}} \right) \frac{\sigma_{\textrm{H}}}{\sigma_{\textrm{NH}}} \
\end{align} where H and NH are the host tree series and the non-host
chronology, in year i, respectively (Swetnam et al. 1985, Swetnam and
Lynch 1989). Only years in which the individual host-tree series and the
non-host chronology overlap are used in Equation 1. The non-host
chronology is scaled by its mean
(\(\overline{\textrm{NH}} \approx 1.0\)) and multiplied by the ratio of
host and non-host standard deviations
(\(\frac{\sigma_{\textrm{H}}}{\sigma_{\textrm{NH}}}\)), which
approximates the variance of the host tree series. This ``corrected''
non-host chronology is subtracted from the host-tree series.

Negative departures in the normalized growth suppression index that
surpass user-defined thresholds in duration and magnitude will be
defined as \emph{defoliation events}.

The individual tree defoliation series are composited in an additional
step to identify \emph{outbreak events} that synchronously affect
mutliple trees. Users have options to define the number and/or the
proportion of trees required for a defoliation event to be considered an
outbreak.

Note that these methods of separating tree- vs site-level disturbance
categories is a major departure from the OUTBREAK program. In OUTBREAK
the two levels of analysis are combined and users have more limited
control of thresholds to define defoliation events versus outbreaks.

\hypertarget{availability-and-installation}{%
\section{Availability and
installation}\label{availability-and-installation}}

The \texttt{dfoliatR} is provided free and open source (Guiterman 2020).
It is provided to \texttt{R} users via the Comprehensive R Archive
Network (CRAN; \url{https://cran.r-project.org/}). To install
\texttt{dfoliatR} from CRAN use

\begin{Shaded}
\begin{Highlighting}[]
\KeywordTok{install.packages}\NormalTok{(}\StringTok{"dfoliatR"}\NormalTok{)}
\end{Highlighting}
\end{Shaded}

In each \texttt{R} session, \texttt{dfoliatR} can be loaded via

\begin{Shaded}
\begin{Highlighting}[]
\KeywordTok{library}\NormalTok{(dfoliatR)}
\end{Highlighting}
\end{Shaded}

Development versions of \texttt{dfoliatR} are available on Github and
installed using the \texttt{devtools} library,

\begin{Shaded}
\begin{Highlighting}[]
\NormalTok{devtools}\OperatorTok{::}\KeywordTok{install_github}\NormalTok{(}\StringTok{"chguiterman/dfoliatR"}\NormalTok{)}
\end{Highlighting}
\end{Shaded}

Issues, bug reports, and ideas for improving \texttt{dfoliatR} can be
posted to \url{https://github.com/chguiterman/dfoliatR/issues}. As an
Open Source library, we welcome and encourage community involvement in
future development. The best ways to contribute to \texttt{dfoliatR} are
through standard Github procedures or by contacting the first author.

\hypertarget{usage}{%
\section{Usage}\label{usage}}

\hypertarget{tree-level-defoliation-events}{%
\subsection{Tree-level defoliation
events}\label{tree-level-defoliation-events}}

\#\#Site-level events

\hypertarget{evaluation}{%
\section{Evaluation}\label{evaluation}}

\hypertarget{extensions}{%
\section{Extensions}\label{extensions}}

\hypertarget{references}{%
\section*{References}\label{references}}
\addcontentsline{toc}{section}{References}

\hypertarget{refs}{}
\leavevmode\hypertarget{ref-Altman2014}{}%
Altman, J., P. Fibich, J. Dolezal, and T. Aakala. 2014. TRADER: A
package for Tree Ring Analysis of Disturbance Events in R.
Dendrochronologia 32:107--112.

\leavevmode\hypertarget{ref-Bunn2008}{}%
Bunn, A. G. 2008. A dendrochronology program library in R (dplR).
Dendrochronologia 26:115--124.

\leavevmode\hypertarget{ref-Bunn2010}{}%
Bunn, A. G. 2010. Statistical and visual crossdating in R using the dplR
library. Dendrochronologia 28:251--258.

\leavevmode\hypertarget{ref-chris_guiterman_2020_3626136}{}%
Guiterman, C. 2020. Chguiterman/dfoliatR: First release. Zenodo.

\leavevmode\hypertarget{ref-Jevsenak2018}{}%
Jevšenak, J., and T. Levanič. 2018. dendroTools: R package for studying
linear and nonlinear responses between tree-rings and daily
environmental data. Dendrochronologia 48:32--39.

\leavevmode\hypertarget{ref-Lara2015}{}%
Lara, W., F. Bravo, and C. A. Sierra. 2015. MeasuRing: An R package to
measure tree-ring widths from scanned images. Dendrochronologia
34:43--50.

\leavevmode\hypertarget{ref-VanderMaaten-Theunissen2015}{}%
Maaten-Theunissen, M. van der, E. van der Maaten, and O. Bouriaud. 2015.
PointRes: An R package to analyze pointer years and components of
resilience. Dendrochronologia 35:34--38.

\leavevmode\hypertarget{ref-Malevich2018}{}%
Malevich, S. B., C. H. Guiterman, and E. Q. Margolis. 2018. burnr: Fire
history analysis and graphics in R. Dendrochronologia 49:9--15.

\leavevmode\hypertarget{ref-Swetnam1989}{}%
Swetnam, T. W., and A. M. Lynch. 1989. A tree-ring reconstruction of
western spruce budworm history in the southern Rocky Mountains. Forest
Science 35:962--986.

\leavevmode\hypertarget{ref-Swetnam1985}{}%
Swetnam, T. W., M. A. Thompson, and E. K. Sutherland. 1985. Using
dendrochronology to measure radial growth of defoliated trees. Page 39p.

\leavevmode\hypertarget{ref-wickham2016ggplot2}{}%
Wickham, H. 2016. Ggplot2: Elegant graphics for data analysis. Springer.

\leavevmode\hypertarget{ref-Wickham2019}{}%
Wickham, H., M. Averick, J. Bryan, W. Chang, L. McGowan, R. François, G.
Grolemund, A. Hayes, L. Henry, J. Hester, M. Kuhn, T. Pedersen, E.
Miller, S. Bache, K. Müller, J. Ooms, D. Robinson, D. Seidel, V. Spinu,
K. Takahashi, D. Vaughan, C. Wilke, K. Woo, and H. Yutani. 2019. Welcome
to the tidyverse. Journal of Open Source Software 4:1686.

\leavevmode\hypertarget{ref-Wickham2020dplyr}{}%
Wickham, H., R. François, L. Henry, and K. Müller. 2020. Dplyr: A
grammar of data manipulation.

\leavevmode\hypertarget{ref-Zang2015}{}%
Zang, C., and F. Biondi. 2015. Treeclim: An R package for the numerical
calibration of proxy-climate relationships. Ecography 38:431--436.


\end{document}


