\documentclass[review]{elsarticle} %review=doublespace preprint=single 5p=2 column
%%% Begin My package additions %%%%%%%%%%%%%%%%%%%
\usepackage[hyphens]{url}

  \journal{Tree-Ring Research} % Sets Journal name


\usepackage{lineno} % add
  \linenumbers % turns line numbering on
\providecommand{\tightlist}{%
  \setlength{\itemsep}{0pt}\setlength{\parskip}{0pt}}

\usepackage{graphicx}
\usepackage{booktabs} % book-quality tables
%%%%%%%%%%%%%%%% end my additions to header

\usepackage[T1]{fontenc}
\usepackage{lmodern}
\usepackage{amssymb,amsmath}
\usepackage{ifxetex,ifluatex}
\usepackage{fixltx2e} % provides \textsubscript
% use upquote if available, for straight quotes in verbatim environments
\IfFileExists{upquote.sty}{\usepackage{upquote}}{}
\ifnum 0\ifxetex 1\fi\ifluatex 1\fi=0 % if pdftex
  \usepackage[utf8]{inputenc}
\else % if luatex or xelatex
  \usepackage{fontspec}
  \ifxetex
    \usepackage{xltxtra,xunicode}
  \fi
  \defaultfontfeatures{Mapping=tex-text,Scale=MatchLowercase}
  \newcommand{\euro}{€}
\fi
% use microtype if available
\IfFileExists{microtype.sty}{\usepackage{microtype}}{}
\bibliographystyle{elsarticle-harv}
\usepackage{color}
\usepackage{fancyvrb}
\newcommand{\VerbBar}{|}
\newcommand{\VERB}{\Verb[commandchars=\\\{\}]}
\DefineVerbatimEnvironment{Highlighting}{Verbatim}{commandchars=\\\{\}}
% Add ',fontsize=\small' for more characters per line
\usepackage{framed}
\definecolor{shadecolor}{RGB}{248,248,248}
\newenvironment{Shaded}{\begin{snugshade}}{\end{snugshade}}
\newcommand{\AlertTok}[1]{\textcolor[rgb]{0.94,0.16,0.16}{#1}}
\newcommand{\AnnotationTok}[1]{\textcolor[rgb]{0.56,0.35,0.01}{\textbf{\textit{#1}}}}
\newcommand{\AttributeTok}[1]{\textcolor[rgb]{0.77,0.63,0.00}{#1}}
\newcommand{\BaseNTok}[1]{\textcolor[rgb]{0.00,0.00,0.81}{#1}}
\newcommand{\BuiltInTok}[1]{#1}
\newcommand{\CharTok}[1]{\textcolor[rgb]{0.31,0.60,0.02}{#1}}
\newcommand{\CommentTok}[1]{\textcolor[rgb]{0.56,0.35,0.01}{\textit{#1}}}
\newcommand{\CommentVarTok}[1]{\textcolor[rgb]{0.56,0.35,0.01}{\textbf{\textit{#1}}}}
\newcommand{\ConstantTok}[1]{\textcolor[rgb]{0.00,0.00,0.00}{#1}}
\newcommand{\ControlFlowTok}[1]{\textcolor[rgb]{0.13,0.29,0.53}{\textbf{#1}}}
\newcommand{\DataTypeTok}[1]{\textcolor[rgb]{0.13,0.29,0.53}{#1}}
\newcommand{\DecValTok}[1]{\textcolor[rgb]{0.00,0.00,0.81}{#1}}
\newcommand{\DocumentationTok}[1]{\textcolor[rgb]{0.56,0.35,0.01}{\textbf{\textit{#1}}}}
\newcommand{\ErrorTok}[1]{\textcolor[rgb]{0.64,0.00,0.00}{\textbf{#1}}}
\newcommand{\ExtensionTok}[1]{#1}
\newcommand{\FloatTok}[1]{\textcolor[rgb]{0.00,0.00,0.81}{#1}}
\newcommand{\FunctionTok}[1]{\textcolor[rgb]{0.00,0.00,0.00}{#1}}
\newcommand{\ImportTok}[1]{#1}
\newcommand{\InformationTok}[1]{\textcolor[rgb]{0.56,0.35,0.01}{\textbf{\textit{#1}}}}
\newcommand{\KeywordTok}[1]{\textcolor[rgb]{0.13,0.29,0.53}{\textbf{#1}}}
\newcommand{\NormalTok}[1]{#1}
\newcommand{\OperatorTok}[1]{\textcolor[rgb]{0.81,0.36,0.00}{\textbf{#1}}}
\newcommand{\OtherTok}[1]{\textcolor[rgb]{0.56,0.35,0.01}{#1}}
\newcommand{\PreprocessorTok}[1]{\textcolor[rgb]{0.56,0.35,0.01}{\textit{#1}}}
\newcommand{\RegionMarkerTok}[1]{#1}
\newcommand{\SpecialCharTok}[1]{\textcolor[rgb]{0.00,0.00,0.00}{#1}}
\newcommand{\SpecialStringTok}[1]{\textcolor[rgb]{0.31,0.60,0.02}{#1}}
\newcommand{\StringTok}[1]{\textcolor[rgb]{0.31,0.60,0.02}{#1}}
\newcommand{\VariableTok}[1]{\textcolor[rgb]{0.00,0.00,0.00}{#1}}
\newcommand{\VerbatimStringTok}[1]{\textcolor[rgb]{0.31,0.60,0.02}{#1}}
\newcommand{\WarningTok}[1]{\textcolor[rgb]{0.56,0.35,0.01}{\textbf{\textit{#1}}}}
\ifxetex
  \usepackage[setpagesize=false, % page size defined by xetex
              unicode=false, % unicode breaks when used with xetex
              xetex]{hyperref}
\else
  \usepackage[unicode=true]{hyperref}
\fi
\hypersetup{breaklinks=true,
            bookmarks=true,
            pdfauthor={},
            pdftitle={dfoliatR: An R package for detection and analysis of insect defoliation signals in tree rings},
            colorlinks=false,
            urlcolor=blue,
            linkcolor=magenta,
            pdfborder={0 0 0}}
\urlstyle{same}  % don't use monospace font for urls

\setcounter{secnumdepth}{0}
% Pandoc toggle for numbering sections (defaults to be off)
\setcounter{secnumdepth}{0}


% Pandoc header



\begin{document}
\begin{frontmatter}

  \title{\emph{dfoliatR}: An R package for detection and analysis of insect
defoliation signals in tree rings}
    \author[a,b]{Christopher H. Guiterman\corref{1}}
   \ead{chguiterman@email.arizona.edu} 
    \author[a,c]{Ann M. Lynch}
   \ead{lyncha@email.arizona.edu} 
    \author[c]{Jodi N. Axelson}
   \ead{jodi.axelson@berkeley.edu} 
      \address[a]{Laboratory of Tree-Ring Research, University of Arizona, 1215 E Lowell
St.~Box 210045, Tucson, AZ, 85721}
    \address[b]{Three Pines Forest Research, LLC, PO Box 225, Etna, NH, 03750}
    \address[c]{U.S. Forest Service, Rocky Mountain Research Station, 1215 E Lowell
St.~Box 210045, Tucson, AZ, 85721}
    \address[d]{Dept of Environmental Science, Policy \& Management, University of
California, Berkeley, Berkeley, CA 94720}
      \cortext[1]{Corresponding Author}
  
  \begin{abstract}
  We present a new R package to provide dendroecologists with tools to
  identify, quantify, analyze, and visualize growth suppression events in
  tree rings produced by insect defoliation. The `dfoliatR' library is
  based on the Fortran V program OUTBREAK, and builds on existing
  resources in the R computing environment. `dfoliatR' expands on OUTBREAK
  to provide greater control of supression thresholds, additional output
  tables, and high-quality graphics. To use `dfoliatR' requires
  standardized ring-width measurements from insect host trees and an
  indexed tree-ring chronology from local non-host trees. It performs an
  indexing procedure to remove the climatic signal represented in the
  non-host chronology from the host-tree series. It then infers
  defoliation events in individual trees. Site-level analyses identify
  outbreak events that synchronously affect a user-defined number or
  proportion of the host trees. Functions are available for summary
  statistics and graphics of tree- and site-level series.
  
  \hfill\break
  \end{abstract}
   \begin{keyword} Dendroecology, spruce budworm, Choristoneura, pandora moth, Coloradia
pandora Blake, larch-bud-moth \newpage\end{keyword}
 \end{frontmatter}

\hypertarget{introduction}{%
\section{Introduction}\label{introduction}}

\hypertarget{overview-of-the-software}{%
\section{Overview of the software}\label{overview-of-the-software}}

The \texttt{dfoliatR} library requires two sets of tree-ring data to
identify defoliation and outbreak events:

\begin{itemize}
\tightlist
\item
  Standardized ring-width series for individual trees of the host
  species
\item
  Standardized tree-ring chronology from a local non-host species
\end{itemize}

Users can develop these data sets in software of their choosing, such as
in \texttt{dplR} or ARSTAN. It is important that the host-tree data
include only one tree-ring series per tree. Both \texttt{dplR} and
ARSTAN have options for averaging mutliple sample series into a
tree-level series. The tree-ring series and chronology can be read into
R via several available \texttt{dplR} functions.

\texttt{dfoliatR} begins to identify defoliation events in individual
trees by removing the climatic signal as represented by the non-host
chronology from the host tree series. This indexing procedure creates a
series -- the ``growth suppression index'' -- in which disturbance is
the predominant signal. Negative departures in the growth suppression
index that surpass user-defined thresholds in duration, and magnitude
will be defined as \emph{defoliation events}. The individual tree
defoliation series are composited in an additional step to identify
\emph{outbreak events} that synchronously affect mutliple trees. Users
have options to define the number and/or the proportion of trees
required for a defoliation event to be considered an outbreak.

Note that these methods of separating tree- vs site-level disturbance
categories is a major departure from the OUTBREAK program. In OUTBREAK
the two levels of analysis are combined and users have more limited
control of thresholds to define defoliation events versus outbreaks.

\hypertarget{availability-and-installation}{%
\section{Availability and
installation}\label{availability-and-installation}}

The \texttt{dfoliatR} is provided free and open source (Guiterman,
2020). It is provided to R users via the Comprehensive R Archive Network
(CRAN; \url{https://cran.r-project.org/}). To install \texttt{dfoliatR}
into R from CRAN use

\begin{Shaded}
\begin{Highlighting}[]
\KeywordTok{install.packages}\NormalTok{(}\StringTok{"dfoliatR"}\NormalTok{)}
\end{Highlighting}
\end{Shaded}

In each R session, \texttt{dfoliatR} can be loaded via

\begin{Shaded}
\begin{Highlighting}[]
\KeywordTok{library}\NormalTok{(dfoliatR)}
\end{Highlighting}
\end{Shaded}

Development versions of \texttt{dfoliatR} are available on Github and
installed using the \texttt{devtools} library in R,

\begin{Shaded}
\begin{Highlighting}[]
\NormalTok{devtools}\OperatorTok{::}\KeywordTok{install_github}\NormalTok{(}\StringTok{"chguiterman/dfoliatR"}\NormalTok{)}
\end{Highlighting}
\end{Shaded}

Support documentation is provided within the package via help menus
(accessed by typing \texttt{?} before a function name), this paper, and
on the package website (\url{https://chguiterman.github.io/dfoliatR/})
which includes updated vignettes. Code to execute a preprint of this
manuscript and the code included below is available at (\emph{insert
later}).

\hypertarget{tree-level-defoliation-events}{%
\section{Tree-level defoliation
events}\label{tree-level-defoliation-events}}

\hypertarget{site-level-events}{%
\section{Site-level events}\label{site-level-events}}

\hypertarget{evaluation}{%
\section{Evaluation}\label{evaluation}}

\hypertarget{extensions}{%
\section{Extensions}\label{extensions}}

\hypertarget{references}{%
\section*{References}\label{references}}
\addcontentsline{toc}{section}{References}

\hypertarget{refs}{}
\leavevmode\hypertarget{ref-chris_guiterman_2020_3626136}{}%
Guiterman, C., 2020. Chguiterman/dfoliatR: First release. Zenodo.
doi:\href{https://doi.org/10.5281/zenodo.3626136}{10.5281/zenodo.3626136}


\end{document}


